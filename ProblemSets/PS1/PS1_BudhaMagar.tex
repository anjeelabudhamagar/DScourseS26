\documentclass{article}



% Useful packages
\usepackage{amsmath}
\usepackage{graphicx}
\usepackage[colorlinks=true, allcolors=blue]{hyperref}

\title{Problem Set 1}
\author{Anjeela Budha Magar}
\date{}
\begin{document}

\maketitle



\section{Introduction}

This course was highly recommended by a PhD student in the OU Economics department. I would like to explore either international trade and policy or migration through societal lenses. I am not sure about the exact framework of the question/s that I would like to work on so far. I am new to coding and using related tools. My goal would be to be comfortable using those tools for efficiency and better communicating my projects. From the classes we have had so far, I can see that it is a new territory for me and I do not want it to be intimidating as it feels right now. I have seen a couple of professors in the Economics department have utilized tools such as R or Github or Stata as tools for teaching effectively. And I would like to be able to create projects and materials that seem to be seamless, clean, and effective forms of communication. My goal is to get into a PhD program in Economics. This class can help me learn and build a base for presenting my ideas.

\section{Equation}
\[a^2+b^2=c^2\]
\end{document}
